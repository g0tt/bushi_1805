\genkoutitle{Ubuntu 18.04 LTS所感}{}{gott}{gott}{}

\section{こんにちは}
16入学のgottです。
最近はLinuxカーネルモジュールを少し触ってみたりHaskellやりたいなって言っていたりします。
過去にもUbuntuの記事を書いているのですが,先日18.04LTSがリリースされ,サークルのサーバに入れてみたので
所感を書いてみます。

ぼくはLTSしか使っていないので16.04との比較が主になると思います。

\section{気になった変更点}
\subsection{デスクトップ環境がGNOMEに}
GNOME派だったので,脱Unityは嬉しいです。モバイル向け?もうAndroidには勝てないでしょ……

\subsection{Kernel 4.15系へ}
ちゃんとSpectre/Meltdownに対応しているらしいです。

\subsection{標準でL2TP over IPSecが利用可能に}
手動でstrongswanとか入れるのが地味に面倒だったのでいいですね。

\subsection{最小構成でのインストールが可能に}
サーバのセットアップ時に最小構成にしてみましたが,本当になにも入ってなくてびっくりしました。GUI使うならServer版よりいいかもしれない。

\subsection{fcitxからibusへ}
Mozcの設定が楽になった気がします。
長らくfcitxだったのでわけもわからずごちゃごちゃ弄っていたら日本語入力できるようになってました。

\subsection{Nautilusのバージョンアップ}
個人的には使いづらくなった気がします。操作が直感的じゃないし,クリック数も増えている気がする……

\subsection{設定画面の変更}
16.04と比べるとツリー構造がかなり変化しているので項目が探しづらいです。

\section{おわりに}
現在個人用のマシンには全て16.04を入れているのですが,18.04にバージョンアップするタイミングを決めかねています。現状特に困っていないですし,最悪あと2年はこのままかもしれないですね……。
18.04のデフォルトには自分好みのパッケージが多いようなので,何らかの理由でクリーンインストールすることがあったら初期設定が多少楽になりそうで嬉しいです。

それではまた。
