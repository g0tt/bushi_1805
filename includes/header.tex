%
% 記事の一般情報(パッケージの読み込みや担当者等々)を記述する
%

%%%%%%%%%%%%%%%%%%%%%%%%%%%%%%%%%%%%%%%%%%%%%%%%%%%%%%%%%%%%%%%%%%%%%%%%%%%
% 使用パッケージ
% 
% 環境によってはコンパイルが通らない場合があります。
% その場合には適宜コメントアウトをしてください。
%%%%%%%%%%%%%%%%%%%%%%%%%%%%%%%%%%%%%%%%%%%%%%%%%%%%%%%%%%%%%%%%%%%%%%%%%%%
\usepackage{amsmath,amssymb,bm,type1cm}
\usepackage{okumacro}
\usepackage{fancybox}
\usepackage{fancyvrb}
\usepackage{longtable}
\usepackage{puyopuyo}
\usepackage{url}
\usepackage{ulem}
\usepackage[yen]{okuverb}
\usepackage{utmcpicins}
\usepackage[dvipdfmx]{graphicx,color,colortbl}
\usepackage{graphicx,url,picins}
\usepackage{verbatim}
\usepackage{longtable}
\usepackage{wrapfig}
\usepackage[scaled]{helvet}
\usepackage[T1]{fontenc}
\usepackage{textcomp}
\usepackage{eepic}
\usepackage{utmcbarcode}
\usepackage{moreverb}
\usepackage{eclbkbox}
\usepackage{framed}
\usepackage{otf}
\usepackage{ascmac}
\usepackage{watermark}
\usepackage{layout}
\usepackage{listings, jlisting}
\usepackage{here}

\pagestyle{headings}

%%%%%%%%%%%%%%%%%%%%%%%%%%%%%%%%%%%%%%%%%%%%%%%%%%%%%%%%%%%%%%%%%%%%%%%%%%%
% 書籍情報の登録
%     by Orios [2008-03-14]
%%%%%%%%%%%%%%%%%%%%%%%%%%%%%%%%%%%%%%%%%%%%%%%%%%%%%%%%%%%%%%%%%%%%%%%%%%%
% 1. 書籍情報
% \volumeinfo{発行年}{月}{日}{分冊番号}{エイリアス}
% ※ 月・日・分冊番号に余計な0をつけない!
%   ○ 4  × 04
\volumeinfo{1}{1}{1}{1}{◯◯部誌}
% 2. 印刷年月日
% \printdate{年}{月}{日}
\printdate{1}{1}{1}
% 3. Cコード・値段
% \pressinfo{Cコード}{値段}
%   [Cコード]
%     9(1ケタ目): 雑誌
%     4(2ケタ目): ムック・その他
%     04(3,4ケタ目): 情報科学
\pressinfo{9404}{0}
% 4. スタッフの名前をここに。 by hyuga [2009-05-01]
\staffinfo{編集担}{絵師たん}

