%
%UTMC部誌 新歓(2008年04月)号用 TeX・登録商標表示ファイル
%
%"shouhyou.tex"
%
%Written by Sword
%Apr. 1,1996
%
%Added by Toyoshima
%July 7,1997
%
%Modified by flatline	2004-04-01
%Modified by nya        2005-01-20
%Modified by Orios	2008-03-09
%<ここから下に本文を書いてね>
%数字やローマ字は出来るだけ半角で書こうね。

~ \vfill % ~ を取り除くと \vfil が無視される。
\noindent
\textbf{各種商標と著作権の取り扱いに関する注意}
\medskip

{\small
\noindent
UNIXは The Open GroupのOS名です。				\\
SolarisはSun Microsystems, Inc.のOS名です。				\\
%Machはカーネギーメロン大学のOS名です。				\\
LinuxはLinus Torvalds氏の手によるOSです。				\\
BSD UNIXとはカリフォルニア大学バークレイ校 (UCB) で開発された
Berkley Software Distribution版に端を発するOS群を指す表現です。				\\
%BeならびにBeロゴは米国Be, Inc.の登録商標です。				\\
%BeBox, BeOS, BeWare, GeekPortは米国Be, Incの商標です。				\\
Apple, Macintosh, MacOSは米国Apple Inc.の登録商標です。				\\
%CP/M, P-CPM, CP/Mupis, CP/M-86, CP/M-68k, CP/M-8000, DR-DOSはデジタルリサーチの、				\\
%PC-DOS, OS/2 WarpはIBMの、				\\
%MSX-DOSはアスキーの、				\\
%OS-9, OS-9/68000, OS-9000, MW CはMICROWAREの、				\\
MS-DOS, Windows95, Windows98, WindowsNT, WindowsMe, Windows2000, WindowsXP, Windows Vista,	\\
MASM (Microsoft Macro Assembler), MS-VB (Microsoft Visual Basic),				\\
MS-C/C++ (Microsoft C/C++), MS-VC/C++/C\# (Microsoft Visual C/C++/C\#)はMicrosoftの、				\\
%Borland C/C++, Turbo C/C++, TASM (Turbo Assembler),				\\
%Turbo Pascal, Delphi, KylixはBorlandの、				\\
CodeGear C++, Turbo C/C++, TASM (Turbo Assembler),				\\
Turbo Pascal, Delphi, KylixはCodeGearの、				\\
N88-BASIC(86)はNECの、				\\
%HuBASICはハドソンソフトの、				\\
Z80はザイログの、				\\
%HuC6270, HuC6280はハドソンの、				\\
8086, i386, i486, Pentium, Pentium Pro, Pentium II, Pentium III, Pentium 4, Pentium M,
Celeron, Core Duo, Core 2 Duo, Core 2 Quad はIntelの、				\\
%MC68000, MC68030, MC68040, MC68060はMOTOROLAの、				\\
PowerPCはIBMの、				\\
それぞれ商標となっております。
その他、プログラム名、プロセッサを始めとする各種の製品名は一般に各メーカーの登録商標です。
本文中では``TM''、並びに``\textregistered''マークは明記しておりませんがその旨ご了承下さい。

本誌に掲載される各記事の著作権は、それぞれの記事の作者が保有します。
各種プログラムのソースコードの著作権はプログラムの作者が保有します。
「PDS」または「パブリックドメイン」と明記されていないものについては、
個人で使用する目的以外のための無断複製は著作権法により禁止されています。
}

%<これより上に本文を書いてね>
%ごくろうさまでした。これでおしまいだよ。〆切に遅れないように出そうね。
